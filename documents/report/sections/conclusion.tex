\paragraph{Findings}
I find it interesting how all of these methods found clearly distinct patterns of genes to distinguish cancer an healthy tissue.
Among the most important genes found very few were common to even 2 of the methods and the vast majority were method specific.
Despite this these genes all seem to be in some way informative enough to allow for accurate classification from all other methods.
It makes me curious to further investigate what about the structure of these models (or quirks in the data or flaws in my work) lead to such stark distinctions.
Cancer is also known to be very complex so the idea that there may be distinct by specific genetic signatures for it is perhaps not too surprising.

\paragraph{Future Directions}
I feel that there is perhaps a way to adapt feature importances as a method of feature engineering/dimensionality reduction.
In a sense we let the machines learn what is important and report back to us, instead of just using them for classification.
It would also be interesting to further investigate the actual genes and pathways discovered here and look for functional consequences or interactions between differently important genes.
There is also the effort by some to develop deep learning methods that still retain some kind of interpretability in thier latent variables \cite{dv}. Associating the latent variables with known functional groups of genes (GO, MSigSB etc) allows research to take advantage of the power of deep learning while still having some sense of what their machine is actually learning.
